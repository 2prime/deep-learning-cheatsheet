\documentclass[pdftex,10pt,a4paper]{scrartcl}

\usepackage[a4paper,left=2.5cm,right=2.5cm,bottom=3cm,top=3cm]{geometry}
\usepackage{amsmath,amssymb,amsthm}
\usepackage{url}
\usepackage[numbers,sort]{natbib}
\parindent=0cm

\title{Graph Autoencoders}
\date{\vspace{-5ex}}

\begin{document}

\maketitle

\section{Introduction}

An autoencoder consists of 3 components: \emph{encoder} $f(x)$, \emph{code} $h$, \emph{decoder} $g(h)$.
The autoenconder compresses the input and produces the code $h = f(x)$, the endocer then reconstructs the input only using this code, so that $g(f(x)) \approx x$.
The learning process is described simply as minimizing a loss function
\begin{equation*}
  L(x, g(f(x))),
\end{equation*}
where $L$ is a loss function penalizing $g(f(x))$ for being dissimilar from $x$ (e.g.\ mean squared error).
\\\\
Autoencoders are mainly a dimensionality reduction (or compression) algorithm.

\begin{itemize}
  \item \textbf{Data-specific:} Autoencoders are only able to meaningfully compress data similiar to what they have been trained on
  \item \textbf{Lossy:} The output of the autoencoder will not be exactly the same as the input
  \item \textbf{Unsupervised:} Autoencoders are considered an \emph{unsupervised learning} technique since they don't need explicit labels to train on.
\end{itemize}

\section{Why?}

\paragraph{Obtaining useful properties}

Copying the input to the output may sound useless, but we are typically not intereset in the output of the decoder.
Instead, we hope that training the autoencoder to perform the input copying task will result in the code taking on useful properties.

\paragraph{Denoising autoencoders}

A \emph{denoising autoencoder} (DAE) instead minimizes
\begin{equation*}
  L(x, g(f(\tilde{x})))
\end{equation*}
where $\tilde{x}$ is a copy of $x$ that has been corrupted by some form of noise.
Denoising autoencoders must therefore undo this operation rather than simply copying their input.

\section{Convolutional autoencoders}

Both the encoder and decoder are usually fully-connected feedforward neural networks and therefore ignore the 2D image or graph structure.
\emph{Convolutional autoencoders} (CAE) differ from conventional autoencoders as their weights are shared among all locations in the input, preserving spatial locality.
Therefore, a convolutional autoencoder needs a way to deconvolute and unpool a convolution respectively pooling operation.
The definition of convolutional autoencoders is relative straightforward on regular grids, but needs several adjustments and generalizations for non-euclidean domains.

\end{document}
